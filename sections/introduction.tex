\section{Introduction} \label{sec:intro}
The electricity grid may be seen as a beautifully decentralized organism: one in which individual energy demands are met by individual generators largely without direct coordination or knowledge of recipient. 
Market signals simply translate into generators commanding their resources to increase or decrease. 
However, as volatile resources like wind and solar replace on-demand resources like most fossil fuels 
-- an important development in the drive to decarbonize the energy supply and tackle climate change -- a potentially worrying question arises: what happens to demand when the generation becomes decoupled from commands; 
i.e., when energy is demanded but the sun isn't shining? 
Grids that do not adequately prepare for this question will face daunting consequences, 
ranging from curtailment of resources \citep{spangher2020prospective} to voltage instability and physical damage.  

A common solution is demand response: a strategy in which customers are incentivized to defer loads to periods of the day where energy is plentiful. 
Given the lack of material infrastructure and cheapness of the incentives, it has several positives above physical energy storage systems. 
% TODO: Look over
One primary application of demand response is in buildings, and both central-level appliance coordination and building-level demand response has been thoroughly studied in residential and industrial settings
(\citep{asadinejad2018evaluation}, \citep{ma2015cooperative}, \citep{li2018integrating}, \citep{yoon2014dynamic}, \citep{johnson2015dynamic}.)
However, while physical infrastructure of office buildings has been studied for demand response (\citep{8248801}), 
there has been no large scale experiment aimed to elicit a behavioral demand response. 

The lack of experiment is perhaps understandable when we consider that the majority of offices do not have a mechanism to pass energy prices onto office workers\citep{das2019novel}. 
If they did, however, not only could a large fleet of decentralized batteries 
-- laptops, cell phone chargers, etc. -- 
be coordinated to function as a large deferable resource, 
but building managers could save money\citep{das2020occupants}.

The SinBerBEST collaboration has developed a Social Game that facilitates workers to engage in competition around energy \citep{konstantakopoulos2019deep}, \citep{konstantakopoulos2019design}. 
Through this framework, a first-of-its-kind experiment has been proposed to implement behavioral demand response within an office building \citep{spangher2020prospective}. 
Prior work has proposed to describe an hourly price-setting controller that learns how to optimize its prices \citep{spangher2020augmenting}. 

However, given the costliness of iterations in this experiment, 
further work in simulation is needed to refine a controller that can adapt the most quickly to the complex behaviors of office workers. 

We endeavor to report one such refinement. 
We will in Section \ref{sec:background} contextualize the architecture of our reinforcement learning controller within the general domain of reinforcement learning 
and introduce a suprise minimizing algorithm that can further improve this controller.
In Section \ref{sec:methods} we will describe the simulation setup and the specific modification we propose. 
In Section \ref{sec:results} we will give results. 
Finally, in Section \ref{sec:discussion} we will discuss implications of the controller and the future work this entails. 
